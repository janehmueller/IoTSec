\documentclass[sigconf]{acmart}

\fancyhf{}

\usepackage{booktabs}
\usepackage{balance}
\usepackage[caption=false]{subfig}
\usepackage[toc,acronym,shortcuts]{glossaries}

\def\pprw{8.5in}
\def\pprh{11in}
\special{papersize=\pprw,\pprh}
\setlength{\paperwidth}{\pprw}
\setlength{\paperheight}{\pprh}
\setlength{\pdfpagewidth}{\pprw}
\setlength{\pdfpageheight}{\pprh}

\newacronym{IoT}{IoT}{Internet of things}

\begin{document}

\title{Title}

\begin{abstract}
%- Mit wenigen Worten das Problem einführen
%- Wichtigste verwandte Arbeit erwähnen
%- Eure wissenschaftlichen Beiträge umreißen
%- Zusammenfassung der Ergebnisse
\end{abstract}

\maketitle

\glsresetall

\section{Introduction}

%- Kontext einführen
%- Problem darstellen
%- Den diesbezüglichen Stand der Technik zusammenfassen
%- Eure wissenschaftlichen Beiträge auflisten

\section{Background}

%- Relevante Protokolle und Konzepte einführen
%- Auf das beschränken, was zum Verstehen der nachfolgenden Abschnitte notwendig ist
%- Bei Platzmangel könnt ihr diesen Abschnitt auch weglassen und stattdessen bei den entsprechenden Stellen Verweise einfügen

\section{Related Work}

%- Umfänglichere Diskussion verwandter Arbeiten
%- Vergleichend mit eurem Ansatz
%- Manchmal bietet es sich an diesen Abschnitt erst vor den Conclusions zu bringen

\section{Approach}

%- Präsentation eures Ansatzes
%- Meistens bietet es sich an erst die grobe Idee zu vermitteln und dann erst ins Detail zu gehen
%- Auf interessante und wichtige Design-Entscheidungen eingehen
%- Modelle einsetzen

\section{Implementation}

%- Überlick über eure Implementierung
%- Modell der Softwarearchitektur

\section{Evaluation}

% Evaluierung interessanter Aspekte anhand von Statistiken

\section{Conclusions and Future Work}

% nochmal auf das Ursprungsproblem und dessen Relevanz zurückkommen
% nochmal zusammenfassen was ihr beigetragen habt
% Schlussfolgerungen ziehen
% Vorschläge für zukünftige Forschungsrichtungen

\balance
\bibliographystyle{ACM-Reference-Format}
\bibliography{bibliography}

\end{document}
