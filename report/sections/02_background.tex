\section{Background}

\begin{enumerate}
	\item Relevante Protokolle und Konzepte einführen
	\item Auf das beschränken, was zum Verstehen der nachfolgenden Abschnitte notwendig ist
	\item Bei Platzmangel könnt ihr diesen Abschnitt auch weglassen und stattdessen bei den entsprechenden Stellen Verweise einfügen
\end{enumerate}

The CoAP protocol defined in RFC7252\footnote{\url{https://tools.ietf.org/html/rfc7252}} can be seen as a lightweight version of HTTP and can be easily mapped on it. Basically, three types of messages can be transferred via UDP. A request can consist of Confirmable (CON) and Non-confirmable (NON) messages. Confirmable messages are expected to be responded by an Acknowledgment (ACK) message. Non-confirmable messages don't require any response but a response can be send as a NON message. All of these message types can use the four methods GET, PUT, POST and DELETE and get back defined status codes that are very similar to their corresponding HTTP methods and codes. Furthermore, every message is identified by an ID and a token and can be adjusted by several options. In \autoref{approach} further specific properties of the CoAP message fields are described and evaluated.

Our target system for the security evaluation was an OpenMote Rev. A1\footnote{\url{www.openmote.com}} which mainly consists of a CC2538 SoC\footnote{\url{www.ti.com/product/CC2538}} running Contiki-NG~\cite{contiki}. As CoAP server implementation target, the Contiki-NG example implementation\footnote{\url{https://github.com/contiki-ng/contiki-ng/blob/develop/examples/coap/coap-example-server/coap-example-server.c}} was used.

On the attacker side, we used the python packet manipulation framework Scapy\footnote{\url{https://scapy.net}} with its latest CoAP community contribution\footnote{\url{https://github.com/secdev/scapy/blob/master/scapy/contrib/coap.py}}. This enables us to easily assemble CoAP messages, perform request and log the response.

% TODO fuzzing techniques
