\documentclass[sigconf]{acmart}

% remove copyright footer
\setcopyright{none}
\settopmatter{printacmref=false}
\renewcommand\footnotetextcopyrightpermission[1]{}

\fancyhf{}

% encoding and language
%\usepackage[english]{babel}
\usepackage[T1]{fontenc}
\usepackage[utf8]{inputenc}

\usepackage{booktabs}
\usepackage{balance}
\usepackage[caption=false]{subfig}
\usepackage[toc,acronym,shortcuts]{glossaries}

\def\pprw{8.5in}
\def\pprh{11in}
\special{papersize=\pprw,\pprh}
\setlength{\paperwidth}{\pprw}
\setlength{\paperheight}{\pprh}
\setlength{\pdfpagewidth}{\pprw}
\setlength{\pdfpageheight}{\pprh}

\newacronym{IoT}{IoT}{Internet of Things}

\begin{document}

\title{Security for the Internet of Things - CoAP Fuzzing}

\author{Jan Ehmueller and Justin Trautmann \\ \today}

\begin{abstract}
%- Mit wenigen Worten das Problem einführen
%- Wichtigste verwandte Arbeit erwähnen
%- Eure wissenschaftlichen Beiträge umreißen
%- Zusammenfassung der Ergebnisse
\end{abstract}

\maketitle

\glsresetall

\section{Introduction}

\begin{enumerate}
	\item Kontext einführen
	\item Problem darstellen
	\item Den diesbezüglichen Stand der Technik zusammenfassen
	\item Eure wissenschaftlichen Beiträge auflisten
\end{enumerate}

\section{Background}
\label{section:background}
%\begin{enumerate}
%	\item Relevante Protokolle und Konzepte einführen
%	\item Auf das beschränken, was zum Verstehen der nachfolgenden Abschnitte notwendig ist
%	\item Bei Platzmangel könnt ihr diesen Abschnitt auch weglassen und stattdessen bei den entsprechenden Stellen Verweise einfügen
%\end{enumerate}

This section first introduces the Constrained Application Protocol and its message format. Then the constrained hardware is presented and finally, multiple fuzzing approaches are explained.

\subsection{CoAP}

The CoAP protocol defined in RFC7252 a lightweight and RESTful protocoll that can be easily mapped on HTTP~\cite{RFC7252}. Basically, four types of messages can be transferred via UDP. A request can consist of Confirmable (CON) and Non-confirmable (NON) messages. Confirmable messages are expected to be responded by an Acknowledgment (ACK) message. Furthermore, reset messages (RST) can be sent. Non-confirmable messages don't require any response but a response can be sent as a NON message. All of these message types can use the four methods GET, PUT, POST and DELETE and get back defined status codes that are very similar to their corresponding HTTP methods and codes. Furthermore, every message is identified by an ID and a token and can be adjusted by several options. In \Autoref{section:approach} we further describe and evaluate specific properties of the CoAP message fields.

\subsection{Hardware}

Our target system for the security evaluation was an OpenMote Rev. A1\footnote{\url{www.openmote.com}} which mainly consists of a CC2538 SoC\footnote{\url{www.ti.com/product/CC2538}} running Contiki-NG~\cite{contiki}. As CoAP server implementation target, the Contiki-NG example implementation\footnote{\url{https://github.com/contiki-ng/contiki-ng/blob/develop/examples/coap/coap-example-server/coap-example-server.c}} was used.

\subsection{Fuzzing}

On the attacker side, we used the python packet manipulation framework \scapy\footnote{\url{https://scapy.net}} with its latest CoAP community contribution\footnote{\url{https://github.com/secdev/scapy/blob/master/scapy/contrib/coap.py}}. This enables us to easily assemble CoAP messages, perform requests and log the responses.

Fuzzing techniques can be categorized with respect to the approach into random fuzzing, mutational fuzzing and generational fuzzing and with respect to the target system into white-box, gray-box and black-box fuzzing~\cite{neystadtPenTesting}.
Random fuzzing is the most inefficient technique with respect to the potentially found errors but relatively simple to set up and best suited for systems that are completely unknown (black-box fuzzing). If some details about the used protocol or message format are known, random fuzzing can be adapted to informed fuzzing so that specific fields such as checksums are still valid in order to avoid instant rejection of the messages.
Mutational fuzzing is based on previously captured valid communication with the system under test that is altered in order to get potentially malicious messages. These messages are still very similar to the usual messages and thus they are not immediately rejected by security mechanisms, which causes them to have a high chance to cause damage.
Lastly generational fuzzing is based on the known structure of the protocol and therefore not suited for black-box fuzzing. Using an empty skeleton of a valid message, fields are randomly or informed randomly filled with values. This makes sure that the message can at least be parsed by the system under test and mechanisms, which are deeper than only the message parser, can be reached and tested.

\section{Related Work}

\begin{enumerate}
	\item Umfänglichere Diskussion verwandter Arbeiten
	\item Vergleichend mit eurem Ansatz
	\item Manchmal bietet es sich an diesen Abschnitt erst vor den Conclusions zu bringen
\end{enumerate}

Previous work on CoAP fuzzing is sparse and only focused on implementation fuzzing. Melo et al.~\cite{Melo2017RobustnessTO} searched for publicly available CoAP implementations on common repository services and search engines and selected a few for testing. Their testing system consisted of the boofuzz framework\footnote{\url{https://github.com/jtpereyda/boofuzz}} in order to start, stop and monitor the systems under test and several \scapy based fuzzing engines. Random as well as mutational and generational fuzzing was applied. As expected, generational fuzzing caused way more crashes than informed random and random fuzzing.

Chen et al.~\cite{chen2018ndss} based their work on the observation that the firmware of most commercial IoT devices is not publicly available but often official smartphone apps are used to communicate with the device. Therefore the app can be used to generate more sophisticated fuzzed messages than random fuzzing would do. They put 17 systems under test that also run different protocols than CoAP and found 8 previously unknown failures.

Furthermore Muench et al.~\cite{EURECOM+5417} addressed the problem of error detection while fuzzing embedded devices in general. Usually total system crashes are used as indicators for successful fuzzing. Unfortunately a variety of software failures such as bugs that cause unintended branching and bring the program in an invalid state or buffer overflows that cause memory corruption and potential execution of data don't cause immediate crashes but lead to general malfunction of the system or cause a crash way later in time. These types of failures are especially hard to find on embedded devices because there are usually just a few mechanisms for memory isolation and protection in place. The authors propose several techniques in order to find these failures even on embedded devices. This includes static instrumentation of the source code or the compiled binary itself with additional memory protection mechanisms, running the software on a more sophisticated host and employ higher level security mechanisms or emulate the device fully or partially. Also hardware instrumentation via debug ports is discussed but rarely applicable on commercial products. Furthermore, different heuristics in order to detect memory corruptions are presented.

Tools for automated code instrumentation in order to detect memory corruptions like the Address Sanitizer~\cite{addressSanitizer} can reliably find memory corruptions when using allocators for memory access. Unfortunately Contiki-NGs memory access is mainly based on global static buffers and therefore cannot be tracked with custom allocators.

Other approaches on fuzzing different layers of the IoT network stack have been made by Böning et al.~\cite{PawelLeo} who targeted the Contiki-NG RPL implementation on fully emulated devices.

Differently to all previous approaches, we tried to apply different fuzzing methods on a CoAP implementation running on a real IoT device in order to find the maximum of also hardware related errors which may be impossible to find in an emulator.
\section{Approach}
\label{section:approach}

\begin{enumerate}
	\item Präsentation eures Ansatzes
	\item Meistens bietet es sich an erst die grobe Idee zu vermitteln und dann erst ins Detail zu gehen
	\item Auf interessante und wichtige Design-Entscheidungen eingehen
	\item Modelle einsetzen
\end{enumerate}

Since we intend to fuzz a CoAP implementation running on a real IoT device we needed to decide on a setup that enables us to properly send messages with a high throughput and to implement a fuzzer in an easy-to-use high-level language. Having these goals we decided against fuzzing as an internal attacker from an embedded device in the IoT network, since it has limited resources compared to a desktop or laptop. It would also mean that we would need to implement the fuzzer in C, which would take us longer and it would be harder to implement features compared to a scripting language like python.

We therefore decided to fuzz the IoT device as an external attacker from a laptop that is not part of the IoT network. To do so we need to be able to communicate with the embedded device via IPv6, which is possible if we set up one OpenMote as a border router and a second one as a CoAP server. Via the border router we can send messages via IPv6 into the IoT network and therefore send messages to the CoAP server. The CoAP server uses the Contiki-NG CoAP implementation and runs its CoAP example server on the application layer. Since we only want to fuzz CoAP it does not matter to us what the server on the application layer does with the payload of the CoAP messages.

Having decided on the setup we chose to implement the fuzzer in python and similarly to Melo et al.~\cite{Melo2017RobustnessTO} we use \scapy to create crafted CoAP packets and send them to the IoT device. Our fuzzing workflow looks as follows. We first send a crafted packet to the CoAP server and await a possible response. After getting the response or getting a timeout (e.g., after sending a NON-message) we then test if the IoT device is still reachable or if it somehow does not respond to messages anymore (e.g., due to a crash or some other malfunction). 

\begin{figure}[h]
		\centering
		TODO
		% \includegraphics[width=0.5\textwidth]{img/coap_header}
		\caption{CoAP header fields defined in the CoAP RFC~\cite{RFC7252}}
		\label{figure:coap_header}
\end{figure}

After determining the message flow we now needed to decide how the craft the CoAP packets. We first read the CoAP RFC and analyzed the CoAP message header in \Autoref{figure:coap_header}. Since we know the protocol being used we chose generational fuzzing as fuzzing approach. And as Contiki-NG is open source we can verify how the header fields are handled and parsed which allows to employ white-box fuzzing. We further describe how we fuzz each header field:
\begin{enumerate}
	\item \textbf{Version}: there is currently only one version of CoAP so the version has to be always 1. Contiki-NG has a simple equality check for this field so there is no point in fuzzing this field.
	\item \textbf{Message Type}: since all four message types are valid we simply randomly fuzz this field.
	\item \textbf{Token Length}: this field contains the length of the token field, which is needed because CoAP has a lot of implicit field sizes in its header. The lengths 9 to 15 are reserved according to the RFC and Contiki-NG tests whether this field is larger than 8~\cite{RFC7252}. We therefore fuzz this field randomly in the range [0, 8].
	\item \textbf{Status Code}: every not implemented status code is simply caught by the implementation. Therefore, we randomly choose either a request or a response status code.
	\item \textbf{Message ID}: this id identifies CoAP messages for the purposed of deduplication and response matching. It has a random value anyway, which means that we do not gain anything by fuzzing this field. But we still need to set it randomly to be able to properly send and receive our crafted CoAP messages.
	\item \textbf{Token}: this field is practically the request id. The length of this field can differ from the token length field, since there are no well-defined boundaries for the following fields. That means it can not be checked if the token is too long or too short. We set this value randomly, but the byte length is independent from the value of the token length field.
	\item \textbf{Options}: CoAP options are parsed by reading the bytes after the token and trying to parse them as an option. These options each have small headers themselves, but unfortunately \scapy does enable us to fuzz these option headers as well.

	After an option was successfully parsed, the next byte is looked at and if it is not the payload marker, the following bytes are parsed as options as well. This goes on until a payload marker is found, which then means that after it, a payload with a non-zero lengths follows.

	We fuzz the CoAP options by randomly selecting a number of options implemented by Contiki-NG, which implements a few more options than those defined in the RFC~\cite{RFC7252}. The values for these options are random strings with a length of 0 to 12.
	\item \textbf{Payload}: the payload goes from the first byte after the payload marker to the last byte of the UDP datagram. Since the payload is not touched by the CoAP engine of Contiki-NG we do not fuzz it.
\end{enumerate}

The final decision we now have to make is how we want to detect if the IoT device crashed or malfunctioned after we sent it a crafted CoAP packet. We discussed the following approaches:
\begin{enumerate}
	\item \textbf{Pinging the device}: the simplest approach to test if a device is still reachable in a network is to ping it. This also works with our IoT devices but since an ICMP echo request is handled on the IP layer, it does not reach the CoAP layer and might be answered even though the CoAP engine somehow malfunctioned.
	\item \textbf{Well-known-core request}: according to the RFC, every CoAP server needs to implement the URI-route ``/.well-known/core``, which describes the resources provided by the server~\cite{RFC7252}. This basically acts as a CoAP ping for us, since it is a route every CoAP server has to provide and to respond to the request the CoAP engine is used. If the content of the response is known, the response after each fuzzing request can be tested for proper structure as well, i.e., that the returned content was not somehow changed by the CoAP engine malfunctioning.
	\item \textbf{Filter CLI output of CoAP server}: we can filter the standard output of the CoAP server for error messages and the log messages of the hardware watchdog, that restarts the device if it froze (e.g., in an infinite loop). We can then add timestamps to the whole standard output and match those with timestamps of the fuzzing requests.
	\item \textbf{Add logging to Contiki-NG}: since Contiki-NG is open source, it is possible to simply add more log output to the CoAP implementation to test, for example, memory content of the global buffers used by Contiki-NG.
	\item \textbf{JTAG debugging}: it is possible to attach a debugger via JTAG to the OpenMotes and read out all the memory. This also enables testing of the validity of the memory and to detect writing into sections of memory that should not be written to by the CoAP implementation.
\end{enumerate}

We decided to test the availability of the IoT device via the Well-known-core request, since it is the most straightforward way to check the reachability of the OpenMote and to check that the CoAP implementation still works.

\section{Implementation}

\begin{enumerate}
	\item Überlick über eure Implementierung
	\item Modell der Softwarearchitektur
\end{enumerate}

\section{Evaluation}

\begin{enumerate}
	\item Evaluierung interessanter Aspekte anhand von Statistiken
\end{enumerate}

When performing fully automated fuzzing, processing of one fuzzed message takes around 5.2 seconds. Setup times for the devices can vary due to cached binaries but can reach the magnitude of several minutes. In order to evaluate the fuzzer and the system under test, we performed 20.000 runs but did not succeeded to trigger a crash. Nonetheless we ensured that potential crashes can be found by our setup by intentionally adding an infinite loop to the CoAP server implementation as reaction to a specific message.  

The scapy tool turned out to be easily usable in order to automatically construct CoAP messages. Even though a random fuzz method is also available in scapy, we decided to fuzz the message fields by ourselves in order to perform informed random or generational fuzzing. Unfortunately, scapy doesn't implement further very common CoAP RFCs. So the observation of CoAP resources like proposed in RFC7641 and the block-wise transfer of larger messages as proposed in RFC7959 is not implemented. Since the system under test does implement these RFCs, this would be a starting point for further research and could be done by either augmenting the scapy CoAP contribution or the fuzzer itself. This also leads to upper bounds with respect to the message size, because CoAP messages should not exceed a IPv6 MTU in order to avoid fragmentation. 


\section{Conclusions and Future Work}
\label{section:conclusion}

%\begin{enumerate}
%	\item nochmal auf das Ursprungsproblem und dessen Relevanz zurückkommen
%	\item nochmal zusammenfassen was ihr beigetragen habt
%	\item Schlussfolgerungen ziehen
%	\item Vorschläge für zukünftige Forschungsrichtungen
%\end{enumerate}

With the goal of building a fully automated fuzzing system for the CoAP implementation of Contiki-NG, we analyzed the implementation and the CoAP protocol itself in order to employ generational fuzzing. With our setup using the Python framework \scapy and the OpenMote hardware, we performed extensive fuzzing but did not succeed in finding serious vulnerabilities in the implementation. This should not be seen as a proof of security and treated carefully since fuzzing can never exhaustively test all possible inputs. Nonetheless, it is an indicator for a sufficiently secure implementation.

\scapy turned out to be easily usable in order to automatically construct CoAP messages. Even though a random fuzz method is also available in \scapy, we decided to fuzz the message fields ourselves. This enables us to perform informed random and generational fuzzing. Unfortunately, \scapy does not implement further CoAP extension RFCs. Such RFCs are the observation of CoAP resources proposed in RFC7641\footnote{\url{https://tools.ietf.org/html/rfc7641}} and block-wise transfers of larger messages proposed in RFC7959\footnote{\url{https://tools.ietf.org/html/rfc7959}}. Since the system under test does implement these RFCs, this would be a starting point for further research and could be done by either extending the \scapy CoAP contribution or the fuzzer itself. This also leads to upper bounds with respect to the message size, because CoAP messages should not exceed an IPv6 MTU in order to avoid fragmentation.

Furthermore, different improvements of the fuzzer can be considered in the future. For example, lots of IoT devices use the payload message format CBOR\footnote{https://tools.ietf.org/html/rfc7049} that can be fuzzed as well. This would enable the fuzzer to not only find vulnerabilities in the protocol parsing and handling, but also in the payload processing logic. 

The logging and replay could be improved as well by providing a feature for selective display of particular messages or errors and a more usable replay of messages. In case of a detected error, automatic replay could be considered.

Since the error detection currently relies on the pure observation of the device output, some errors might have happened, but could not be detected by our system. Such errors include any kind of memory corruption that can potentially lead to malfunctioning in the future, but cannot be detected from the outside at the moment of occurrence. In order to detect them immediately, the implementation or the device has to be additionally instrumented, which makes the fuzzer no longer generic to all devices and implementations but could also be considered.


% visually balance both columns
\balance

\bibliographystyle{ACM-Reference-Format}
\bibliography{bibliography}

\end{document}
